\maketitle
\pagestyle{plain}
\chapter*{前言}
大学物理(University Physics)是本校理工科学生在大一、大二年级所要学习的一门自然科学
基础课程。这门课程课时较多、内容丰富,相关的练习题与考试题则尤显花样繁多,充分考验着每一个
学生对相关知识的掌握程度与应用能力。从掌握知识的角度来说,多做、精做大物习题是学好这门
课程的必经之路;从备考、应试的角度来说,若不熟练掌握各类大物习题的思路与解法,而仅依靠课内所
学到的基本知识点,则不可能在考试中取得令人满意的成绩。因此,熟练掌握本课程相关练习题的
解题技巧,是非常必要的。

“不积跬步,无以至千里。”一份可靠的题解,需要经过多次的改进才可望真正铸造出来。虽然这份题解的确可谓“精心制作”,但笔误、错漏等在所难免,特别需要各位使用者帮助我们指正。如您在参考的过程中发现有任何错误
之处,欢迎您通过下面的方式联系我们,帮助我们改进这份题解:
%\begin{itemize}
	%\item \faGithub ~~ GitHub平台论坛(\textbf{推荐,但需要注册}):\url{https://github.com/qyxf/BookHub/issues}
	%\item \faInternetExplorer ~~ 钱院学辅信息发布站:\url{https://qyxf.github.io}
	%\item \faEnvelopeOpen ~~ 钱院学辅邮箱:\texttt{qianxiaofu.mail@qq.com}
	%\item \faQq ~~ 钱院学辅官方答疑墙:~~\textbf{钱小辅}~~206713407
%\end{itemize}

作为钱院学辅出品的第一份“重量级”作品,希望它能够带给每一位同学最好的体验!

\begin{flushright}
	钱院学辅大物编写小组\\
	2019 年 12 月 7 日
\end{flushright}
\vspace{1.0cm}
\begin{figure}[!h]
	\centering
	\begin{minipage}[c]{0.5\textwidth}
		\centering
		\includegraphics[scale=0.5]{./template/qrcode2.png}
	\end{minipage}%
	\begin{minipage}[c]{0.5\textwidth}
		\centering
		\includegraphics[scale=0.5]{./template/new_group.png}
	\end{minipage}
\end{figure}
%需要交流分享群二维码


\cleardoublepage
\tableofcontents