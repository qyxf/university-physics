\maketitle
\pagestyle{plain}
\chapter*{下册序}
大学物理(University Physics)是本校理工科学生在大一、大二年级所要学习的一门自然科学
基础课程。目前,大多数大学物理的课堂均以一套统一印制的活页练习题作为课下作业。这套题目按章布置,
每章均有选择题、填空题、计算题三个部分,题量适中,覆盖了各章所有较为重要的知识点,并能
够使同学们充分地将课上所学知识用于实际问题的解决过程中。基于种种原因,这份题目的答案未见
公开;这能够保证大多数同学独立地完成作业,但不利于大家检查错误、在参考过程中发现自己的
问题。

为了解决这一问题,自2019年3月以来,钱院学辅组织了一些正在学习本课程的同学,编写了这份大学物理题解。历时九个多月,这两学期布置的所有作业之题解都已编写完成,每道题均有较详细的分析与求解过程可供参考。全书始终采用\LaTeX 整理,这使得本题解的排版效果得到了充分的保证。我们希望,这份精心制作的题解,能够给正在学习与将要学习这门课程的同学提供充分的帮助,使他们能够更好的掌握这项重要的基础课之内容。

大学物理下册主要涉及三大模块:热力学和统计物理学基础、振动与波及波动光学基础、量子物理和固体物理基础,它们之间相互独立,各成体系。相比于上册,这些内容更需要对物理规律深刻的认识,而习题、考试只是帮助我们更好地掌握所学内容。下册的作业题也并不像上册的力学那样需要很多技巧,更多关注的是对所学的理解。希望大家在进行期末复习时,能对照课本和讲义,抓住知识主线,深入思考,并以作业题为线索对物理规律的应用有所把握。

本学期期末考试将采取第三方命题的形式。一般来说,作业题的难度都要比考试高,所以大家大可不必担心,做好自己的复习,沉着冷静地面对考试,就能取得理想的成绩。

钱院学辅是一个开放的平台,欢迎全校同学积极参与讨论。由于本题解刚刚编辑完成,如果您在使用过程中发现有笔误、错漏,亦或是对某题的解析有自己的想法,期待更深入的分析和讨论,欢迎您通过下面的方式联系我们,帮助我们改进这份题解:
\begin{itemize}
	\item \faGithub ~~ GitHub平台论坛(\textbf{推荐,但需要注册}):\url{https://github.com/qyxf/BookHub/issues}
	\item \faInternetExplorer ~~ 钱院学辅信息发布站:\url{https://qyxf.github.io}
	\item \faEnvelopeOpen ~~ 钱院学辅邮箱:\texttt{qianyuanxuefu@163.com}
	\item \faQq ~~ 钱院学辅交流分享群:~~\texttt{群号:}~~852768981
\end{itemize}
我们会及时处理。

作为钱院学辅出品的第一份“重量级”作品,希望它能够带给每一位同学最好的体验!

\begin{flushright}
	钱院学辅大物编写小组\\
	2019 年 12 月 17 日
\end{flushright}
\vspace{1.0cm}
\begin{figure}[!h]
	\centering
	\begin{minipage}[c]{0.4\textwidth}
		\centering
		\includegraphics[width=\linewidth]{./template/qrcode2.png}
	\end{minipage}%
	\begin{minipage}[c]{0.4\textwidth}
		\centering
		\includegraphics[width=\textwidth]{./template/new_group.png}
	\end{minipage}
\end{figure}

\cleardoublepage
\tableofcontents
