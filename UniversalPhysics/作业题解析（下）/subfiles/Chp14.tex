\chapter{机械波}

\section{选择题}

\exercise C

\solve 由波动方程的式子$y=0.08cos(10\pi t-4\pi x)$(SI)

知圆频率$\omega=10\pi$,频率$f=\frac{\omega}{2 \pi}=5Hz$;

知$2\pi\lambda=4\pi$,得波长$\lambda=0.5m$,得波速$u=\lambda f=2.5m/s$。

根据以上结果排除选项,知答案为D。

\exercise D

\solve 

选项A:$x$前系数为负,说明波正向传播,错误;

选项B:将方程化为标准形式:$y=Acos(a(bt-x)-\phi)$,可以看出$x$前系数为负,说明波正向传播,错误;

选项C:由波动方程知,在$x$轴上有些点永远不会振动(是驻波方程),不是行波,错误;

选项D:方程有$y=Acos(ax+t)+Acos(ax+t-\phi)$,则每一个位置的振动都可以看成是两个等振幅等圆频率的振动的合成,一定可以合成为$y=Acos(ax+t+\phi_2)$的形式;而合成振动的$x$前的系数为正,则是负向传播,正确。

\exercise A

\solve 将$t=0.5s$带入$y=0.20cos[2\pi (t-x/2)+\pi]$得$y(x,0.5)=0.20cos(\pi x)$,由图像可知选A。

\exercise D

\solve 由图像的最高点可知振幅$A=\sqrt{2}m$,由周期$T=4s$,波长$\lambda=4m$,及波沿x轴正向传播,得波函数为$y=\sqrt{2}cos[2\pi({\frac{t}{T}-\frac{x}{\lambda}})+\phi_0]=\sqrt{2}cos(\frac{\pi}{2}(t-x)+\phi_0)$。则$y(0,t)=\sqrt{2}cos(\frac{\pi}{2}t+\phi_0)$,由图中可知$\sqrt{2}cos\phi_0=-\frac{\sqrt{2}}{2}$,又由此时$x=0$处质点在$t=0$处即将向下运动知,$\phi_0=\frac{2}{3}\pi$。则有$y=\sqrt{2}cos(\frac{\pi}{2}(t-x)+\frac{2}{3}\pi)$,选D。

\exercise C

\solve  由图中可得振幅$A=0.01$m,波长$\lambda=200$m,频率$f=\frac{u}{\lambda}=1$Hz,周期$T=\frac{1}{f}=1\mathrm{s}$。由于周期为$1\mathrm{s}$,则$t=1\mathrm{s}$和$t=0\mathrm{s}$的振动情况是一样的。设P的振动方程为$y=0.01cos(2\pi\frac{t}{T}+\phi_0)=0.01cos(2\pi t+\phi_0)$。而在$t=0\mathrm{s}$时,有$cos(\phi_0)=0.5$,且P点将向下运动,则$\phi_0=\frac{1}{3}\pi$。

综上所述,P点的运动方程为$y=0.01cos(2\pi t+\frac{1}{3}\pi)$,选C。

\exercise D

\solve  设在$r$处,波的单位面积能量为$B=kI$,则以波源为中心,以$r$为半径上的球面的总能量为$E=4\pi r^2 kI$。而由能量守恒知,不同球面上波的总能量是相同的,所以有$E=C$(常量),则$I=\frac{C}{4\pi k r^2}$。则有$I\propto \frac{1}{r^2}$,选D。

\exercise D

\solve 由图中位置关系可知,$S_1$到达P点比$S_2$到达P点超前$\frac{\lambda}{2}$。则$S_1$和$S_2$的相位差为$\phi_1-\phi_2=\frac{ 2\pi\lambda}{2 \lambda}+\frac{\pi}{2}=\frac{3\pi}{2}$,选D。 

\exercise A

\solve 两列波到达P点的相位差为$\Delta\phi=\frac{2\pi\Delta x}{\lambda}=\frac{2\pi\Delta x f}{u}$。

由于在P点相消干涉,则$\Delta\phi=(2k+1)\pi$。

有$f=\frac{u}{2\Delta x}(2k+1)=\frac{172}{1.3}(2k+1)$。

为使$1350\leqslant f \leqslant 1826$,取k=6,有$f=1720Hz$,选A。

\exercise D

\solve 弦上产生的驻波的频率要满足$u=\frac{u}{2f}k,k=1,2\ldots$。则不能产生任意频率,排除AB。

在微小横振动时,质元的势能$E_p\propto (\frac{\partial y}{\partial x}|_x)^2$,即质元所在位置的弦的斜率的平方越大,此处势能越大。则在弦上各点达到最大位移时,在波节处的质元斜率平方最大,则在波节处质元的势能最大,选D。

\exercise D

\solve 由多普勒效应,在波源与观察者相向而行时,有$f=\frac{u+v_1}{u-v_2}f_0$,其中$v_1$是观察者即火车的速度,$v_2$是波源即汽车的速度,$u$是空气中声速。计算得$f=1.2548kHz$,最接近为D选项,选D。

\section{填空题}
\exercise $1.2\mathrm{m}$\qquad$0.1\mathrm{m}$

\solve 波长$\lambda=uT=1.2\mathrm{m}$;

由于两点在波的传播方向上,则相位差有$\Delta \phi=\frac{2\pi \Delta x}{\lambda}$,则计算得$\Delta x=0.1\mathrm{m}$。

\exercise $\frac{2\pi}{k}\qquad Acos(\omega t+\pi)\qquad -\frac{1}{2}\frac{\rho \omega^3}{k}A^2$

\solve 由于机械波向$x$轴负向传播,则有$k=\frac{2\pi}{\lambda}$,得到$\lambda=\frac{2\pi}{k}$;

将$x=\frac{\lambda}{2}=\frac{\pi}{k}$代入,得$y(\frac{\pi}{k},t)=Acos(\omega t+k\frac{\pi}{k})=Acos(\omega t+\pi)$;

平均能流密度为$\textbf{I}=\bar{w}\textbf{u}$。而波的平均能量密度为$\bar{w}=\frac{1}{2}\rho A^2 \omega^2$,又知波沿x轴负方向传播,故$u=-\frac{\omega}{k}$。则$I=-\frac{1}{2}\frac{\rho \omega^3}{k}A^2$。

\exercise 0.6$\qquad$30

\solve 相位差与间距有关系$\Delta \phi=\frac{2\pi \Delta x}{\lambda}$,则有$\lambda=\frac{2\pi \Delta x}{\Delta \phi}=0.6\mathrm{m}$,波速$u=\lambda f=30\mathrm{m/s}$。

\exercise 减小

\solve 由于A处质元的弹性势能在减小,则此时A向上运动,则A的速度减小,则振动动能减小。

\exercise $40\qquad D,E \qquad -20$

\solve 此时A点为两个波峰叠加,高度为20cm,B点为两个波谷叠加,高度为-20cm,则A,B两点的高度差为40cm,且A,B两点均为振动加强的点;

由于D,E点在此时均由波峰和波谷相遇合成,故D,E点为振动减弱点,而C点在A,B所连线段中间,故也是振动加强点;

由于C点是振动加强点,则C点的振幅为20cm,在此时刻C点高度为0,且下一时刻向下运动,故此时C点的相位为$\frac{\pi}{2}$;由$\omega=\frac{2\pi u}{\lambda}=10\pi\ \mathrm{rad/s}$,得经过$\Delta t=0.65\mathrm{s}$后,相位增加$\Delta \phi=\omega\Delta t=6.5\pi$,则C点的振动相位为$\phi=0.5\pi+6.5\pi=7\pi$,C点为波谷,故高度(位移)为-20cm。

\exercise 不同$\qquad$相同

\solve 驻波方程为$y=Acos(\frac{2\pi x}{\lambda})cos(\frac{2\pi t}{T}+\phi_0)$,则在相邻波节之间,各点的振幅为$|Acos(\frac{2\pi x}{\lambda})|$,振幅不相同;同时相邻波节之间$Acos(\frac{2\pi x}{\lambda})$总是同正负,则相位总是相同。

\exercise \pi

\solve 由驻波方程可知,在$x_1$处的振动为$y_1=\frac{\sqrt{2}}{2}Acos(15\pi t)$,在$x_2$处的振动为$y_2=-\frac{\sqrt{2}}{2}Acos(15\pi t)=\frac{\sqrt{2}}{2}Acos(15\pi t+\pi)$。则两个振动的相位相差\pi。

\exercise 光疏介质$\qquad$光密介质

\solve 根据定义,一列光波从一种介质向另一种介质入射,光速较大的介质叫做光疏介质,光速较小的叫做光密介质。

\exercise 朝向$\qquad$0.25

\solve 设空气中声速为$u_0$,设声源朝向观察者的速度是$u$。则观察者接收到的波长$\lambda=\frac{u_0-u}{u_0}\lambda_0=\frac{3\lambda_0}{4}$,解出$u=\frac{1}{4}u_0=0.25u_0$。由于解出u大于0,则声源朝向观察者运动,且运动速度为空气中声速的0.25倍。

\exercise 8.48m/s

\solve 设潜艇移动的速度为$u$,则在潜艇接收到的信号频率为$f_1=\frac{u_0+u}{u_0}f_0$,则潜艇反射的信号频率为$f=\frac{u_0}{u_0-u}f_1=\frac{u_0+u}{u_0-u}f_0$。可知探测器接收到的信号频率增大,则$f-f_0=341$,得$f=30341Hz$。已知$u_0=1500\mathrm{m/s}$,代入计算得$u=8.48\mathrm{m/s}$。


\section{计算题}
\exercise

\solve (1)由于波向x正向传播,则在$x$处的质点,振动的相位比$0.1$m处的质点落后$\Delta \phi(x)=\frac{\omega}{u}(x-0.1)$。而在0.1m处的质点振动为$y_0=0.5sin(1.0-4.0t)=0.5cos(4t-1+\frac{\pi}{2})$。则可得圆频率$\omega=4\mathrm{rad/s}$,则$\Delta \phi(x)=5(x-0.1)$。

由以上可得$y(x,t)=0.5cos(4t-1+0.5\pi-5(x-0.1))=0.5cos(4t-5x+1.07)\mathrm{m}$

(2)$v(0.1,t)=\dy{y(0.1,t)}{t}=-2sin(4t+0.57)\mathrm{m/s}$

(3)$v_{max}=2\mathrm{m/s}$;则$\frac{v_{max}}{u}=5:2=2.5$。

\exercise 

\solve 由坐标变换,有
\[
\begin{cases}
	x'=-(x-\frac{\lambda}{4})\\
	y'=y
\end{cases}
\]
即
\[
\begin{cases}
	x=\frac{\lambda}{4}-x'\\
	y=y'
\end{cases}
\]

将上式代入波动方程$y=Acos(2\pi(\frac{t}{T}-\frac{x}{\lambda})+\varphi)$,得新坐标下的波动方程为:

$$y'=Acos(2\pi(\frac{t}{T}+\frac{x'}{\lambda})+\varphi-\frac{\pi}{2})$$

\exercise

\solve (1)$f=\frac{1}{T}=2Hz$,$u=\lambda f=1.6\mathrm{m/s}$;(或$u=\frac{\lambda}{T}=1.6\mathrm{m/s}$)

(2)$y(x,t)=0.2cos(2\pi(\frac{t}{0.5}-\frac{x}{0.8})+\varphi_0)\mathrm{m}$;

而$y(0.2,t)=0.2cos(-\frac{\pi}{2}+\varphi_0)$,由$t=0$时,$x=0.2m$的质点处于反向最大位移处,则有$-\frac{\pi}{2}+\varphi_0=\pi$,则$\varphi_0=\frac{3\pi}{2}$。

则$y(x,t)=0.2cos(2\pi(\frac{t}{0.5}-\frac{x}{0.8})+\frac{3\pi}{2})\mathrm{m}$;

(3)$y(\frac{3}{4}\lambda,t)=0.2cos(4\pi t)\mathrm{m}$;

(4)$\Delta x=0.3\mathrm{m}$,$\Delta \varphi=\frac{\Delta x}{\lambda}2\pi=0.75\pi$。

\exercise

\solve (1)$y_1=Acos(2\pi(\frac{t}{T}-\frac{x}{\lambda})+\frac{3\pi}{2})\mathrm{m}$;

(2)反射波方向与入射波相反,但是振幅,频率和波速不变,故设反射波方程为$y_2=Acos(2\pi(\frac{t}{T}+\frac{x}{\lambda})+\frac{3\pi}{2}-\Delta \varphi)\mathrm{m}$。

而反射波到达原点时,相位比入射波落后$\Delta \varphi =2\times\frac{3}{4}\lambda\times \frac{2\pi}{\lambda}+\pi=4\pi$(其中最后加的$\pi$是半波损失的相位)。而$4\pi$是$2\pi$的整数倍,故不影响方程的形式。则有$y_2=Acos(2\pi(\frac{t}{T}+\frac{x}{\lambda})+\frac{3\pi}{2})\mathrm{m}$;

(3)$y=y_1+y_2=2Acos(\frac{2\pi x}{\lambda})cos(\frac{2\pi t}{T}+\frac{3\pi}{2})\mathrm{m}$

则当$cos(\frac{2\pi x}{\lambda})=0$时为波节,则$\frac{2\pi x}{\lambda}=\frac{2n+1}{2}\pi$,得$x=\frac{2n+1}{4}\lambda$,其中$n$是整数。则在图中$x=\frac{1}{4}\lambda,\frac{3}{4}\lambda$处标注为静止点。